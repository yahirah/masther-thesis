\documentclass[10pt,a4paper]{article}
\usepackage{amsmath,graphicx,subfigure}

\title{{Master Thesis\\[0.5em]}
       {\bf \huge Development of a game for hand- and eye coordination in children rehabilitation.\\[0.5em]}
       {\bf Project Plan (version 2)}}
\author{Anna Maria Walach, S121540}
\setlength{\parindent}{0mm}
\setlength{\parskip}{\medskipamount}

\begin{document}

\maketitle

\section{Overall Project Plan}
%proritize healthy children for tests

The plan counts weeks starting from Monday, 31st August.

\begin{tabular}{c|l|c}
%  \multicolumn{2}{c|}{} & \multicolumn{2}{c|}{MRF} & \multicolumn{2}{c|}{Poisson} & \multicolumn{2}{c}{MPU}\\ 
%  \# &   Points & Acc & HF & Acc & HF & Acc & HF \\
  Week  & Activity                                            						& Risk
  \\
  \hline
  1-2   & Cerebral palsy - literature study, preparing project documentation     	& 1 \\
  3-4	& Current solutions - literature study	   									& 1 \\
  5-8	& Finding \& analysing available commercial hardware 			 			& 1 \\
   9    & Preparing to implementation (organizing devices, licenses, environment)   & 2 \\
  10-12	& Preparing test scenarios and game design, i.e. control schema,       		& 2 \\
  12-18 & Prototyping									   							& 3 \\
  16-20 & Testing on people														   	& 3 \\
  18-23 & Analysis of the data and finalizing the report     												& 1 \\
  \hline
%  \multicolumn{2}{l|}{Average}  & 0.13  &  0.90 &   0.42 &   1.46 &   0.12 &   1.73\\
%  \hline
\end{tabular}
  
\subsection{Risk Analysis}

In the plan the risk is classified using a scale from 1 (no risk) to 5
(high risk). The risk is described as the chance of the activity being
delayed.

The preparation phase has low risk, that is caused by expected difficulties in getting access to devices. It can be reduced by focusing on devices owned by DTU, Helene Elsass Center or friends, with publicly available SDKs. 

The game design phase in weeks 10-12 has low risk, but based on the feedback it may get delayed. The solution to reduce risk is to confront the assumptions and propositions with advisers and therapists as often as possible, to avoid developing whole design based on wrong assumptions. 

The implementation tasks (prototyping and adjusting prototypes) has an average high risk. That is caused by using new frameworks. It is being reduced by getting to know the libraries before starting the development phase and choosing the tools with excellent documentation and large community. 

The high risk is assigned to the testing phase. Subjects may not be available at that time or in the number that allows for meaningful testing. The risk can be reduced by adjusting test scenarios to let healthy children be also a meaningful part of test group, i.e. focusing test scenarios on activities measurement, not the effects. Also, informing test subjects in advance about expected date of testing can help in reducing the risk as well.
\end{document}

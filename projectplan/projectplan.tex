\documentclass[10pt,a4paper]{article}
\usepackage{amsmath,graphicx,subfigure}

\title{{Master Thesis\\[0.5em]}
       {\bf \huge Development of a game for hand- and eye coordination in children rehabilitation.\\[0.5em]}
       {\bf Project Plan (version 1)}}
\author{Anna Maria Walach, S121540}
\setlength{\parindent}{0mm}
\setlength{\parskip}{\medskipamount}

\begin{document}

\maketitle

\section{Overall Project Plan}


The plan counts weeks starting from Monday, 31st August.

\begin{tabular}{c|l|c}
%  \multicolumn{2}{c|}{} & \multicolumn{2}{c|}{MRF} & \multicolumn{2}{c|}{Poisson} & \multicolumn{2}{c}{MPU}\\ 
%  \# &   Points & Acc & HF & Acc & HF & Acc & HF \\
  Week  & Activity                                            						& Risk
  \\
  \hline
  1-2   & Cerebral palsy - literature study, preparing project documentation     	& 1 \\
  3-4	& Current solutions - literature study	   									& 1 \\
  5-8	& Finding \& analysing available commercial hardware and software 			& 1 \\
   9    & Preparing to implementation (organizing devices, licenses, environment)   & 2 \\
  10-14	& Prototyping       														& 3 \\
  12-16 & Testing with subjects from Helene Elsass Center							& 5 \\
  14-20 & Adjusting prototypes, preparing new tests, analysing tests results.   	& 3 \\
  18-23 & Finalizing the report     												& 1 \\
  \hline
%  \multicolumn{2}{l|}{Average}  & 0.13  &  0.90 &   0.42 &   1.46 &   0.12 &   1.73\\
%  \hline
\end{tabular}
  
\subsection{Risk Analysis}

In the plan the risk is classified using a scale from 1 (no risk) to 5
(high risk). The risk is described as the chance of the activity being
delayed.

The implementation tasks (preparing, prototyping and adjusting prototypes) has a low risk. It is caused by using new frameworks and libraries for software and hardware, that may be a reason of some unexpected delay in implementation or getting the hardware and licences. However, it should not exceed a week or two of delay and can be reduced by using well-documented and established frameworks as well as solutions already bought by DTU, which should be become criteria in the initial analysis.

The highest risk is assigned to the testing phase. Subjects may not be available at that time or in the number that allows for meaningful testing. The risk can be reduced by trying to aim the game for less severe cases of the cerebral palsy and test it with available, healthy children and also informing the Center, children and their parents far in advance about planned tests session.
\end{document}

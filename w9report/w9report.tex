\documentclass[10pt,a4paper]{article}
\usepackage{amsmath,graphicx,subfigure, hyperref, url}

\title{{Master Thesis\\[0.5em]}
       {\bf \huge Development of a game for hand- and eye coordination in children rehabilitation.\\[0.5em]}
       {\bf Weekly Report 9}}
\author{Anna Maria Walach, S121540}
\setlength{\parindent}{0mm}
\setlength{\parskip}{\medskipamount}

\begin{document}

\maketitle

\section*{What has been done this week}
During last week, I applied MoSCoW method to the analysis I performed to determine priority of criteria:
\begin{itemize}
\item Must have
\begin{itemize}
\item detailed gesture recognition supported out-of-the-box
\item price not bigger than 300\$
\item acceptable documentation and any kind of support for Unity
\item high comfort of usage
\item be available for customer use
\end{itemize}
\item Should have
\begin{itemize}
\item excellent documentation
\item support for Unity provided by the manufacturer
\item worldwide availability 
\item community significant in size
\end{itemize}
\item Could have
\begin{itemize}
\item physical feedback
\item price lower than 200\$
\end{itemize}
\item Won't have
\end{itemize}

Here is a list of tested hand motion-tracking devices, with short summary of the analysis:

\textbf{Leap Motion}

Leap Motion is a camera-based device dedicated to hand, or rather palm tracking. The device has excellent documentation, explicit support for Unity, low price and is available worldwide. Unfortunately, it does not support any form of gesture recognition and because of small field of view is very inconvenient to use. 

\textbf{Play Station Move}

Move is wand-like controller for Play Station 3, used together with a camera for tracking hand movement. The device offers acceptable documentation, explicit support for Unity and very low price, while being available worldwide, but does not provide any kind of detailed palm tracking, let alone gesture recognition. 

\textbf{Wii Remote}

The device very similar to PS Move, although in this case documentation and SDK is available upon registration and acceptance from the Nintendo, so it was not possible to estimate the quality of it. Like Move, it does not offer any kind of detailed hand tracking.

\textbf{Sphero 2.0}

Sphero is a ball with ability to move and track its movement and rotation, while also providing physical feedback(shaking). It offer acceptable documentation and support for Unity (although only for mobiles) and is available worldwide for reasonable price. It does not offer gesture recognition.

\textbf{Xbox Kinect}

This most popular commercial device for motion tracking offers excellent documentation and Unity support for acceptable price. It is available worldwide. It does support hand tracking, but struggle with hand recognition - it is designed to track the movement of whole body. Microsoft research the possibilities of incorporating advance gesture recognition in Kinect software \cite{handpose}, but they'll not be available for customers for next few years. 

\textbf{Myo}

Myo is an armband, detecting the movement of the hand as well as gestures. It offers excellent documentation, Unity support and is available worldwide. It supports detection of five gestures and can track the movement without any delay. 

Because Myo sensors uses muscle activity, I contacted the therapist and consulted the possibility of using Myo as a device for rehabilitation. It should work well for patients with GMFCS score I-III, which is similar to expected target group.

\textbf{Intel RealSense}

RealSense is another camera-based solution, but with software dedicated to voice, face and hand recognition. It has excellent documentation and support for Unity. It is available worldwide, also built-in laptops, tablets and similar devices. It can recognize 14 different gestures.


I think that Myo and RealSense are two most promising devices. Unfortunately, I don't have latest generation of RealSense available right now, so I prepared two different plans:

\textbf{Plan A} \\
1. Ordering Real Sense now. \\ 
2. Developing and testing prototype with Myo (Nov, Dec). \\
3. Developing and testing second prototype with RealSense (Dec,January).\\

\textbf{Plan B} \\ 
1. Just developing and testing with Myo only.

The reason why I would like to go for Plan A is that it would allow me to directly compare camera-based and sensor-based solution, as well as be able to have a back-up plan if after all the Myo won't work well with affected hands. 
%order real sense now
%develop the one for myo
%during christmas develop one for real sense

\section*{Project status according to the study plan}
There is no delay in the plan.

\section*{Plan for the next weeks}
Designing and developing the game.

\bibliographystyle{plainurl}
\bibliography{../thesis/bibliography/Bibliography}


\end{document}
